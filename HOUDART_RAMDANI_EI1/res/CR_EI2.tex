% Document : Compte rendu GRASP et ReactiveGRASP pour le SPP
% Auteur : Xavier Houdart et Djibril Ramdani
% Année académique : 2024-2025


% Paramètres de mise en page
\setlength{\topmargin}{-25mm}
\setlength{\textheight}{250mm}

% Configuration pour le code Julia
\lstdefinelanguage{Julia}{
    keywords={abstract,break,case,catch,const,continue,do,else,elseif,end,export,false,for,function,immutable,import,importall,if,in,macro,module,otherwise,quote,return,switch,true,try,type,typealias,using,while},
    sensitive=true,
    alsoother={\$},
    morecomment=[l]\#,
    morecomment=[n]{\#=}{=\#},
    morestring=[b]',
    morestring=[b]"
}
\lstset{
    language=Julia,
    basicstyle=\ttfamily\small,
    numbers=left,
    numberstyle=\tiny\color{gray},
    keywordstyle=\color{blue},
    commentstyle=\color{green},
    stringstyle=\color{red},
    frame=single,
    breaklines=true,
    tabsize=4
}

\begin{minipage}

\title{\textbf{GRASP et ReactiveGRASP pour le Set Packing Problem (SPP)}}
\author{Xavier Houdart et Djibril Ramdani}
\date{Année académique : 2024-2025}
\maketitle

\section*{Mise en Place de GRASP}
L'algorithme GRASP (\textit{Greedy Randomized Adaptive Search Procedure}) utilise deux phases principales pour générer une solution optimisée pour le SPP :
\begin{enumerate}
    \item \textbf{Construction d'une solution initiale :} Génération d'une solution faisable avec un processus glouton randomisé.
    \item \textbf{Recherche locale :} Optimisation de la solution initiale en explorant son voisinage immédiat.
\end{enumerate}

\subsection*{Fonction \texttt{constructionSolution(C, A, alpha)}}
Cette fonction crée une solution réalisable initiale pour le SPP en utilisant $\alpha$ pour ajouter de la variabilité au processus glouton.

\begin{lstlisting}
function constructionSolution(C, A, alpha)
    m, n = size(A)
    solution = zeros(Int, n)
    contraintes = zeros(Int, m)
    ensembleDeSolutions = []
    pasFini = true
    score = 0
    cAutre = Float16[]
    while pasFini
        ensembleDeSolutions = []
        for i = 1:n
            vraiFaux = verifieBon(C, A, contraintes, i, solution)
            if vraiFaux
                push!(ensembleDeSolutions, [compteCDiviseParnb1(C, A, i), i])
            end
        end
        if isempty(ensembleDeSolutions)
            pasFini = false
        else
            ensembleDeSolutions = sort!(ensembleDeSolutions)
            nbAGarder = size(ensembleDeSolutions)[1] - Int(round(alpha * size(ensembleDeSolutions)[1]))
            if nbAGarder == 0
                nbAGarder = 1
            end
            ensembleDeSolutions = last(ensembleDeSolutions, nbAGarder)
            nb = rand(1:size(ensembleDeSolutions)[1])
            sol = ensembleDeSolutions[nb]
            for i = 1:m
                contraintes[i] += A[size(A)[1] * (Int(sol[2]) - 1) + i]
            end
            solution[Int(sol[2])] = 1
        end
    end
    return solution, score
end
\end{lstlisting}

\subsection*{Fonction \texttt{grasp(C, A, tour, alpha)}}
Cette fonction utilise \texttt{constructionSolution} pour générer une solution initiale, puis applique une descente locale pour l'améliorer.

\begin{lstlisting}
function grasp(C, A, tour, alpha)
    bestSol = 0
    bestScore = 0
    for i = 1:tour
        solution, score = constructionSolution(C, A, alpha)
        solution, score = descente(C, A, solution, score)
        if score > bestScore
            bestScore = score
            bestSol = solution
        end
    end
    return bestScore, bestSol
end
\end{lstlisting}

\section*{Ajout du Composant ReactiveGRASP}
ReactiveGRASP adapte dynamiquement $\alpha$ en fonction des performances obtenues au cours des itérations, améliorant ainsi la capacité de GRASP à explorer l’espace de solutions.

\begin{lstlisting}
function reactiveGRASP(C, A, pas, Nalpha, tours)
    nbTbl = Int(floor(1 / pas))
    tblAlpha = []
    tblAvg = []
    tblqDek = []
    for i = pas:pas:1
        push!(tblAlpha, [i, 1 / nbTbl])
        push!(tblAvg, [i, 0, 0])
    end
    sBest = []
    zBest = 0
    zWorst = 10^10
    for i = 1:nbTbl
        score, solution = grasp(C, A, 1, tblAlpha[i][1])
        if score > zBest
            zBest = copy(score)
            sBest = copy(solution)
        end
        if score < zWorst
            zWorst = copy(score)
        end
        tblAvg[i][2] += score
        tblAvg[i][3] += 1
    end
    for i = 1:tours
        for j = 1:Nalpha
            alpha = nbAlea(tblAlpha)
            score, solution = grasp(C, A, 1, tblAlpha[alpha][1])
            if score > zBest
                zBest = copy(score)
                sBest = copy(solution)
            end
            if score < zWorst
                zWorst = copy(score)
            end
            tblAvg[alpha][2] += score
            tblAvg[alpha][3] += 1
        end
        SommeqDei = 0
        for j = 1:size(tblAlpha)[1]
            zAvg = tblAvg[j][2] / tblAvg[j][3]
            push!(tblqDek, ((zAvg - zWorst) / (zBest - zWorst)))
            SommeqDei += ((zAvg - zWorst) / (zBest - zWorst))
        end
        for j = 1:size(tblAlpha)[1]
            pDek = tblqDek[j] / SommeqDei
            tblAlpha[j][2] = max(pDek, 0.000001)
        end
    end
    return zBest, sBest
end
\end{lstlisting}


\section{Tableau récapitulatif des résultats}

\begin{longtable}{|l|l|c|c|c|}
\hline
\textbf{Fichier} & \textbf{Méthode} & \textbf{Temps (secondes)} & \textbf{Z (Score)} & \textbf{GLPK (Z)} \\
\hline
\endfirsthead

\hline
\textbf{Fichier} & \textbf{Méthode} & \textbf{Temps (secondes)} & \textbf{Z (Score)} & \textbf{GLPK (Z)} \\
\hline
\endhead

\hline
\multicolumn{5}{|r|}{\textit{Suite de la page suivante}} \\
\hline
\endfoot

\hline
\endlastfoot

didacticSPP.dat & Glouton + descente & 0.0437 & 30 & 30.0 \\
                & GRASP (20,0.75) & 0.0005 & 30 & \\
                & ReactiveGRASP (0.05,15,20) & 0.0086 & 30 & \\
                & Avec GLPK & 0.0010 & - & 30.0 \\
\hline
pb\_1000rnd0400.dat & Glouton + descente & 0.0912 & 40 & 330.26 \\
                    & GRASP (20,0.75) & 2.3715 & 59 & \\
                    & ReactiveGRASP (0.05,15,20) & 36.5139 & 67 & \\
                    & Avec GLPK & 2.0207 & - & 330.26 \\
\hline
pb\_100rnd0200.dat & Glouton + descente & 0.0269 & 30 & 50.0 \\
                   & GRASP (20,0.75) & 0.5767 & 31 & \\
                   & ReactiveGRASP (0.05,15,20) & 8.5205 & 32 & \\
                   & Avec GLPK & 0.0038 & - & 50.0 \\
\hline
pb\_200rnd0100.dat & Glouton + descente & 0.1967 & 365 & 581.07 \\
                   & GRASP (20,0.75) & 5.6167 & 403 & \\
                   & ReactiveGRASP (0.05,15,20) & 116.6724 & 414 & \\
                   & Avec GLPK & 0.0229 & - & 581.07 \\
\hline
pb\_200rnd0400.dat & Glouton + descente & 0.3119 & 57 & 100.0 \\
                   & GRASP (20,0.75) & 5.8933 & 58 & \\
                   & ReactiveGRASP (0.05,15,20) & 91.5351 & 59 & \\
                   & Avec GLPK & 0.0089 & - & 100.0 \\
\hline
pb\_200rnd1000.dat & Glouton + descente & 0.1580 & 115 & 118.0 \\
                   & GRASP (20,0.75) & 3.2338 & 115 & \\
                   & ReactiveGRASP (0.05,15,20) & 51.5594 & 116 & \\
                   & Avec GLPK & 0.0027 & - & 118.0 \\
\hline
pb\_200rnd1700.dat & Glouton + descente & 0.0648 & 223 & 350.57 \\
                   & GRASP (20,0.75) & 1.0452 & 233 & \\
                   & ReactiveGRASP (0.05,15,20) & 20.7520 & 243 & \\
                   & Avec GLPK & 0.0141 & - & 350.57 \\
\hline
pb\_500rnd0500.dat & Glouton + descente & 0.0683 & 84 & 373.30 \\
                   & GRASP (20,0.75) & 1.7989 & 117 & \\
                   & ReactiveGRASP (0.05,15,20) & 31.5817 & 115 & \\
                   & Avec GLPK & 0.2473 & - & 373.30 \\
\hline
pb\_500rnd0600.dat & Glouton + descente & 0.0841 & 6 & 28.31 \\
                   & GRASP (20,0.75) & 1.4999 & 7 & \\
                   & ReactiveGRASP (0.05,15,20) & 23.0839 & 7 & \\
                   & Avec GLPK & 0.7815 & - & 28.31 \\
\hline
pb\_500rnd1800.dat & Glouton + descente & 0.0818 & 9 & 31.27 \\
                   & GRASP (20,0.75) & 1.6614 & 11 & \\
                   & ReactiveGRASP (0.05,15,20) & 24.0966 & 12 & \\
                   & Avec GLPK & 0.3786 & - & 31.27 \\
\hline
\end{longtable}


\end{minipage}
