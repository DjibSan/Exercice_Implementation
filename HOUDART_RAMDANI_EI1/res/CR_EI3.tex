
\section*{Livrable de l'exercice d'implémentation  3 : \\ Algorithme Génétique pour le SPP}

%
% -----------------------------------------------------------------------------------------------------------------------------------------------------
%



\begin{minipage}

\section*{Question 1}
Pour votre problème de SPP, mettre en place une seconde métaheuristique (autre que celle déployée dans les EI précédents) parmi les métaheuristiques autres que GRASP/VNS/ILS (SA, TS, GA, ACO...) abordées en cours.

Dans cette question, nous avons choisi de développer une métaheuristique basée sur un algorithme génétique (\textit{Genetic Algorithm}, GA) pour résoudre le problème du \textit{Set Packing Problem} (SPP). L'algorithme génétique est une technique d'optimisation inspirée des principes de la sélection naturelle et de l'évolution biologique. Cette méthode repose sur les concepts de population, sélection, croisement (\textit{crossover}) et mutation.

\section*{Description de l'algorithme}

L'algorithme génétique proposé pour le SPP suit les étapes suivantes :

\subsection*{1. Création de la population initiale}

Nous avons utilisé la métaheuristique \textit{ReactiveGRASP} pour générer les solutions initiales. Le meilleur paramètre $\alpha$ trouvé par \textit{ReactiveGRASP} est utilisé pour guider la distribution des solutions. La population initiale est composée de :
\begin{itemize}
    \item 100 solutions générées en ajustant $\alpha$ autour de sa meilleure valeur avec des variations prédéfinies (par exemple, $\pm 0.05$, $\pm 0.1$, etc.).
    \item 50 solutions générées aléatoirement avec des valeurs de $\alpha$ comprises entre 0 et 1.
\end{itemize}

\subsection*{2. Évaluation des solutions}

Chaque solution est évaluée en fonction de son score (la qualité de la solution pour le SPP). Les solutions sont ensuite triées par ordre décroissant de score, et un processus de \textbf{normalisation des scores} est appliqué pour calculer leur aptitude (\textit{fitness}) relative. Cette aptitude est utilisée pour la sélection des parents.

\subsection*{3. Sélection des parents}

Une technique de sélection probabiliste est utilisée, où les solutions ayant un meilleur score ont une probabilité plus élevée d'être sélectionnées comme parents pour la génération suivante. Ce mécanisme favorise l'exploitation des meilleures solutions tout en maintenant la diversité génétique.

\subsection*{4. Croisement (\textit{Crossover})}

Pour chaque paire de parents sélectionnés, un croisement est effectué en combinant les parties des solutions parentales. Des points de croisement aléatoires sont choisis pour déterminer quelles parties des parents sont échangées pour produire des enfants.

\subsection*{5. Mutation}

Une étape de mutation est appliquée aux enfants générés pour introduire de la diversité dans la population. La probabilité de mutation diminue exponentiellement en fonction du nombre de mutations appliquées :
\begin{itemize}
    \item 5\% de chance pour une mutation simple.
    \item 0,25\% de chance pour deux mutations simultanées.
    \item Des probabilités encore plus faibles pour trois ou quatre mutations.
\end{itemize}

\subsection*{6. Validation des solutions}

Après le croisement et la mutation, les solutions sont validées pour garantir qu'elles respectent les contraintes du SPP. Les solutions invalides sont ajustées ou supprimées.

\subsection*{7. Itérations}

Les étapes 2 à 6 sont répétées pour un nombre prédéfini de générations (ou jusqu'à atteindre une limite de temps) afin d'améliorer progressivement la qualité des solutions.

\section{Tableau récapitulatif des résultats}

\begin{longtable}{|l|l|c|c|c|}
\hline
\textbf{Fichier} & \textbf{Méthode} & \textbf{Temps (s)} & \textbf{Z (Score)} & \textbf{GLPK (Z)} \\
\hline
\endfirsthead

\hline
\textbf{Fichier} & \textbf{Méthode} & \textbf{Temps (s)} & \textbf{Z (Score)} & \textbf{GLPK (Z)} \\
\hline
\endhead

\hline
\multicolumn{5}{|r|}{\textit{Suite de la page suivante}} \\
\hline
\endfoot

\hline
\endlastfoot

didacticSPP.dat & Glouton + descente & 0.0437 & 30 & 30.0 \\
                & GRASP (20, 0.75) & 0.0005 & 30 & \\
                & ReactiveGRASP (0.05, 15, 20) & 0.0086 & 30 & \\
                & Avec GLPK & 0.0010 & - & 30.0 \\
                & Algorithme Génétique & 0.7607 & 53 & \\
\hline
pb\_1000rnd0400.dat & Glouton + descente & 0.0912 & 40 & 330.26 \\
                    & GRASP (20, 0.75) & 2.3715 & 59 & \\
                    & ReactiveGRASP (0.05, 15, 20) & 36.5139 & 67 & \\
                    & Avec GLPK & 2.0207 & - & 330.26 \\
                    & Algorithme Génétique & 60.3847 & 56 & \\
\hline
pb\_100rnd0200.dat & Glouton + descente & 0.0269 & 30 & 50.0 \\
                   & GRASP (20, 0.75) & 0.5767 & 31 & \\
                   & ReactiveGRASP (0.05, 15, 20) & 8.5205 & 32 & \\
                   & Avec GLPK & 0.0038 & - & 50.0 \\
                   & Algorithme Génétique & 11.4961 & 33 & \\
\hline
pb\_200rnd0100.dat & Glouton + descente & 0.1967 & 365 & 581.07 \\
                   & GRASP (20, 0.75) & 5.6167 & 403 & \\
                   & ReactiveGRASP (0.05, 15, 20) & 116.6724 & 414 & \\
                   & Avec GLPK & 0.0229 & - & 581.07 \\
                   & Algorithme Génétique & 62.6872 & 403 & \\
\hline
pb\_200rnd0400.dat & Glouton + descente & 0.3119 & 57 & 100.0 \\
                   & GRASP (20, 0.75) & 5.8933 & 58 & \\
                   & ReactiveGRASP (0.05, 15, 20) & 91.5351 & 59 & \\
                   & Avec GLPK & 0.0089 & - & 100.0 \\
                   & Algorithme Génétique & 68.2101 & 59 & \\
\hline
pb\_200rnd1000.dat & Glouton + descente & 0.1580 & 115 & 118.0 \\
                   & GRASP (20, 0.75) & 3.2338 & 115 & \\
                   & ReactiveGRASP (0.05, 15, 20) & 51.5594 & 116 & \\
                   & Avec GLPK & 0.0027 & - & 118.0 \\
                   & Algorithme Génétique & 42.4782 & 115 & \\
\hline
pb\_200rnd1700.dat & Glouton + descente & 0.0648 & 223 & 350.57 \\
                   & GRASP (20, 0.75) & 1.0452 & 233 & \\
                   & ReactiveGRASP (0.05, 15, 20) & 20.7520 & 243 & \\
                   & Avec GLPK & 0.0141 & - & 350.57 \\
                   & Algorithme Génétique & 19.1723 & 237 & \\
\hline
pb\_500rnd0500.dat & Glouton + descente & 0.0683 & 84 & 373.30 \\
                   & GRASP (20, 0.75) & 1.7989 & 117 & \\
                   & ReactiveGRASP (0.05, 15, 20) & 31.5817 & 115 & \\
                   & Avec GLPK & 0.2473 & - & 373.30 \\
                   & Algorithme Génétique & 38.0390 & 103 & \\
\hline
pb\_500rnd0600.dat & Glouton + descente & 0.0841 & 6 & 28.31 \\
                   & GRASP (20, 0.75) & 1.4999 & 7 & \\
                   & ReactiveGRASP (0.05, 15, 20) & 23.0839 & 7 & \\
                   & Avec GLPK & 0.7815 & - & 28.31 \\
                   & Algorithme Génétique & 32.2313 & 7 & \\
\hline
pb\_500rnd1800.dat & Glouton + descente & 0.0818 & 9 & 31.27 \\
                   & GRASP (20, 0.75) & 1.6614 & 11 & \\
                   & ReactiveGRASP (0.05, 15, 20) & 24.0966 & 12 & \\
                   & Avec GLPK & 0.3786 & - & 31.27 \\
                   & Algorithme Génétique & 28.2061 & 12 & \\
\hline

\end{longtable}


\section*{Analyse des Métaheuristiques}

Pour répondre à la question d’évaluation des deux métaheuristiques (ReactiveGRASP et l’algorithme génétique), nous analysons leurs performances en utilisant les indicateurs pertinents extraits du tableau récapitulatif des résultats : le temps d’exécution (\textbf{Temps (s)}) et la qualité des solutions (\textbf{Z (Score)}).

\subsection*{2.1 Analyse en faveur de ReactiveGRASP}

\begin{itemize}
    \item \textbf{Qualité des solutions (\textbf{Z})} : ReactiveGRASP obtient systématiquement des scores proches ou supérieurs à ceux des autres méthodes, notamment le glouton et GRASP simple. Par exemple, pour le fichier \texttt{pb\_200rnd0100.dat}, ReactiveGRASP atteint un score de 414, supérieur à celui de l’algorithme génétique (403) et des autres approches. De plus, dans le cas de \texttt{pb\_200rnd1700.dat}, il atteint 243, contre 237 pour l’algorithme génétique.
    \item \textbf{Fiabilité des solutions} : ReactiveGRASP produit des solutions robustes avec des scores élevés, particulièrement sur des cas complexes. Par exemple, pour \texttt{pb\_500rnd0500.dat}, il atteint un score de 115, proche des meilleures solutions.
\end{itemize}

\subsection*{2.2 Analyse en défaveur de ReactiveGRASP}

\begin{itemize}
    \item \textbf{Temps d’exécution (\textbf{Temps (s)})} : ReactiveGRASP est souvent la méthode la plus lente. Par exemple, sur \texttt{pb\_500rnd0500.dat}, son temps d’exécution est de 31.58 secondes, ce qui est significativement plus lent que les 38.04 secondes de l’algorithme génétique. Dans les instances de grande taille, cette lenteur est un inconvénient majeur.
\end{itemize}

\subsection*{2.3 Analyse en faveur de l’algorithme génétique}

\begin{itemize}
    \item \textbf{Temps d’exécution (\textbf{Temps (s)})} : L’algorithme génétique est généralement plus rapide que ReactiveGRASP pour des instances complexes. Par exemple, sur \texttt{pb\_200rnd1000.dat}, il s’exécute en 42.48 secondes contre 51.56 secondes pour ReactiveGRASP.
    \item \textbf{Qualité compétitive des solutions (\textbf{Z})} : L’algorithme génétique offre souvent des scores compétitifs, voire supérieurs dans certains cas. Par exemple, pour \texttt{pb\_500rnd0500.dat}, il atteint un score de 103, ce qui est supérieur à GRASP (117) et proche de ReactiveGRASP (115).
\end{itemize}

\subsection*{2.4 Analyse en défaveur de l’algorithme génétique}

\begin{itemize}
    \item \textbf{Qualité des solutions (\textbf{Z})} : Bien que compétitif, l’algorithme génétique produit parfois des solutions inférieures. Par exemple, pour \texttt{pb\_200rnd1700.dat}, il atteint un score de 237 contre 243 pour ReactiveGRASP.
\end{itemize}

\subsection*{2.5 Conclusion}

Les deux métaheuristiques présentent des forces et des faiblesses. ReactiveGRASP se distingue par la qualité des solutions générées, mais son temps d’exécution est un frein pour des instances complexes. En revanche, l’algorithme génétique offre un bon compromis entre la qualité des solutions et le temps d’exécution, le rendant plus adapté dans des contextes nécessitant une réponse rapide. Le choix entre ces deux approches dépendra donc des priorités : qualité optimale ou rapidité.




\end{minipage}
